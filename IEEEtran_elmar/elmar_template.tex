\documentclass[a4paper,conference]{IEEEtran}

% this template contains common rules used for conference paper on ELMAR symposium therefore explanation of used packages/commands are not included in template
\usepackage{ifpdf}
\usepackage{cite}
\usepackage[pdftex]{graphicx}
\usepackage{array}
\usepackage{mdwmath}
\usepackage{mdwtab}
\usepackage{amssymb,latexsym}
\usepackage{stfloats}
\usepackage{amsmath}
\usepackage{subfig}

\DeclareRobustCommand*{\IEEEauthorrefmark}[1]{%
\raisebox{0pt}[0pt][0pt]{\textsuperscript{\footnotesize\ensuremath{#1}}}}

\hyphenation{op-tical net-works semi-conduc-tor}

\renewcommand\IEEEkeywordsname{Keywords}

\renewcommand{\citedash}{--} 

\begin{document}

\title{Template for ELMAR Conference Paper} %capitalize each word
% The template is designed so that author affiliations are not repeated each time for multiple authors of the same affiliation. Please keep your affiliations as succinct as possible (for example, do not differentiate among departments of the same organization). 
\author{\IEEEauthorblockN{
First Author\IEEEauthorrefmark{1},
Second Author\IEEEauthorrefmark{2},
Third Author\IEEEauthorrefmark{2}}
\IEEEauthorblockA{\IEEEauthorrefmark{1}
Organization/University\\
Adress, Country}
\IEEEauthorblockA{\IEEEauthorrefmark{2}
Organization/University\\
Adress, Country}
{\it CorrespondingAuthor@gmail.com}}

\maketitle

\begin{abstract}
This electronic document is a "live" template. The various components of your paper [title, text, heads, etc.] are already defined.
\end{abstract}

\begin{keywords}
Component; Formatting; Style; Styling; Insert
\end{keywords}

\IEEEpeerreviewmaketitle

\section{Introduction}
\label{sec1}
This template is modified form of IEEE Latex template for conferences papers and should be used only for paper on ELMAR symposium. It provides authors with most of the formatting specifications needed for preparing electronic versions of their papers. All standard paper components have been specified for three reasons: (1) ease of use when formatting individual papers, (2) automatic compliance to electronic requirements that facilitate the concurrent or later production of electronic products, and (3) conformity of style throughout a conference proceedings.

Margins, column widths, line spacing, and type styles are built-in; examples of the type styles are provided throughout this document. Some components are not prescribed, although the various text styles and simple examples are provided. The formatter will need to create other components, incorporating the applicable criteria that follow.

This paper is organized as follows. In Section \ref{sec2} common rules and styles are defined. Section \ref{sec3} describes language issues and rules.

\section{Common style rules}
\label{sec2}
This template has been tailored for output on the \textbf{A4 paper size}%avoid using bold text
. You are not allowed to use US letter-sized paper for IWSSIP symposium paper.
Also, paper should not be longer than \textbf{four (4) pages} including references, tables, figures and/or appendix.

\subsection{Styling}
\label{subsec1}
All headings, margins, column width, etc. are predefined so authors should just use proper style as shown in this template.

Headings, or heads, are organizational devices that guide the reader through your paper. There are two types: component heads and text heads.
Component heads identify the different components of your paper and are not topically subordinate to each other. Examples include Acknowledgments and References. 

Text heads organize the topics on a relational, hierarchical basis. For example, the paper title is the primary text head because all subsequent material relates and elaborates on this one topic. If there are two or more sub-topics, the next level head should be used and, conversely, if there are not at least two sub-topics, then no subheads should be introduced. 

\subsection{Figures and Tables}
\label{subsec}
Place figures and tables at the top and bottom of columns. Avoid placing them in the middle of columns. Large figures and tables may span across both columns. Figure captions should be below the figures; table heads should appear above the tables. Insert figures and tables after they are cited in the text. Use the abbreviation "Fig. \ref{figure_example}", even at the beginning of a sentence.

\begin{table}[!b]
% increase table row spacing, adjust to taste
\renewcommand{\arraystretch}{1.3}
\caption{\textsc{An Example of a Table}}
\label{table_example}
\centering
% Some packages, such as MDW tools, offer better commands for making tables
% than the plain LaTeX2e tabular which is used here.
\begin{tabular}{|c||c|}
\hline
\textbf{One} & \textbf{Two}\\
\hline
A & 1\\
\hline
B & 2\\
\hline
\end{tabular}
\end{table}

\begin{figure}[!b]
\center
\includegraphics{fig1.png}
\caption{An example of a figure}
\label{figure_example}
\end{figure}

%example of more complex figure style
\begin{figure}[b!]
\center
\subfloat[First image]{\label{sd1}
\includegraphics[width=0.21\textwidth]{fig1.png}}\hspace{0.5em}
\subfloat[Second image]{\label{sd2}
\includegraphics[width=0.21\textwidth]{fig1.png}}\\
\vspace{0.5em}
\subfloat[Third image]{\label{sd3}
\includegraphics[width=0.21\textwidth]{fig1.png}}\hspace{0.5em}
\subfloat[Fourth image]{\label{sd4}
\includegraphics[width=0.21\textwidth]{fig1.png}}
\caption{Example of using subfigures}
\label{subfigure_example}
\end{figure}

Figure Labels: Use 8 point Times New Roman for Figure labels. Use words rather than symbols or abbreviations when writing Figure axis labels to avoid confusing the reader. As an example, write the quantity "Magnetization", or "Magnetization, M", not just "M". If including units in the label, present them within parentheses. Do not label axes only with units. In the example, write "Magnetization (A/m)" or "Magnetization {A[m(1)]}", not just "A/m". Do not label axes with a ratio of quantities and units. For example, write "Temperature (K)", not "Temperature/K".

\subsection{Equations}
\label{subsec3}
This subsection defines some common types of equations through few examples:

\begin{enumerate} % this is an example of ordered lists
\item The LBP of neighbourhood {\it P} and radius {\it R} \cite{ref3} is obtained by thresholding the values of neighbourhood pixels ${\it g_p}$ using the value of the central pixel ${\it g_c}$: % every used variable should be explained and every equation should be numbered
\begin{eqnarray}
LPB(P,R) = \sum\limits_{p=0}^{P-1}{s(g_p-g_c)\times2^p}, \\
s(x) = \left\{
\begin{array}{lr}
1, \quad x \ge 0,\\
0, \quad \text{otherwise.}
\end{array}
\right.
\label{eq:lbp}
\end{eqnarray}
\item Multiline equation: % long equations should be broken in to lines
\begin{multline}
N(p_c) = N(p_{x,y}) = \{p_{x+i,y-1}\}, \\ i = (-\lfloor k/2 \rfloor,...,\lfloor k/2 \rfloor) \qquad
\label{eq:neigh}
\end{multline}
\item Special equations:
\begin{equation}
b_i = \left\{
\begin{array}{lr}
1, \quad p_i \ge mean(N(p_c) \cup p_c)\\
0, \quad text{otherwise}
\end{array}
\right.
\label{eq:mean}
\end{equation}
\end{enumerate}

Use "\eqref{eq:lbp}", not "Eq. \eqref{eq:lbp}" or "equation ", except at the beginning of a sentence "Equation "\eqref{eq:lbp} is . . ."

\subsection{Citation}
\label{subsec}
All references should be cited in text: \cite{ref1}, \cite{ref2}. This is an example of multiple references: \cite{ref1, ref2, ref3, ref4}.

\section{Language}
\label{sec3}

\subsection{Abbreviations and Acronyms}
\label{subsec2}
Define abbreviations and acronyms the first time they are used in the text, even after they have been defined in the abstract. Abbreviations such as IEEE, SI, MKS, CGS, sc, dc, and rms do not have to be defined. Do not use abbreviations in the title or heads unless they are unavoidable.

\subsection{Units}
\label{subsec2}
Please follow these rules: %this is an example of bulleted list
\begin{itemize}
\item Use either SI (MKS) or CGS as primary units. (SI units are encouraged.) English units may be used as secondary units (in parentheses). An exception would be the use of English units as identifiers in trade, such as "3.5-inch disk drive".
\item Avoid combining SI and CGS units, such as current in amperes and magnetic field in oersteds. This often leads to confusion because equations do not balance dimensionally. If you must use mixed units, clearly state the units for each quantity that you use in an equation.
\item Do not mix complete spellings and abbreviations of units: "Wb/m\textsuperscript{2}" or "webers per square meter", not "webers/m\textsuperscript{2}". Spell out units when they appear in text: ". . . a few henries", not ". . . a few H".
\item Use a zero before decimal points: "0.25", not ".25". Use "cm\textsuperscript{3}", not "cc".
\end{itemize}

\subsection{Some Common Mistakes}
\label{subsec4}
\begin{itemize}
\item The word "data" is plural, not singular.
\item The subscript for the permeability of vacuum 0, and other common scientific constants, is zero with subscript formatting, not a lowercase letter "o".
\item In American English, commas, semi-/colons, periods, question and exclamation marks are located within quotation marks only when a complete thought or name is cited, such as a title or full quotation. When quotation marks are used, instead of a bold or italic typeface, to highlight a word or phrase, punctuation should appear outside of the quotation marks. A parenthetical phrase or statement at the end of a sentence is punctuated outside of the closing parenthesis (like this). (A parenthetical sentence is punctuated within the parentheses.)
\item A graph within a graph is an "inset", not an "insert". The word alternatively is preferred to the word "alternately" (unless you really mean something that alternates).
\item Do not use the word "essentially" to mean "approximately" or "effectively".
\item In your paper title, if the words "that uses" can accurately replace the word "using", capitalize the "u"; if not, keep using lower-cased.
\item Be aware of the different meanings of the homophones "affect" and "effect", "complement" and "compliment", "discreet" and "discrete", "principal" and "principle".
\item Do not confuse "imply" and "infer".
\item The prefix "non" is not a word; it should be joined to the word it modifies, usually without a hyphen.
\item There is no period after the "et" in the Latin abbreviation "et al.".
\item The abbreviation "i.e." means "that is", and the abbreviation "e.g." means "for example".
\end{itemize}

\section*{Acknowledgment}

The authors would like to thank...

% Refer simply to the reference number, as in [3]—do not use "Ref. [3]" or "reference [3]" except at the beginning of a sentence: "Reference [3] was the first . . ." Unless there are six authors or more give all authors' names; do not use "et al.". Papers that have not been published, even if they have been submitted for publication, should be cited as "unpublished" [4]. Papers that have been accepted for publication should be cited as "in press" [5]. Capitalize only the first word in a paper title, except for proper nouns and element symbols. For papers published in translation journals, please give the English citation first, followed by the original foreign-language citation [6].
% All references should be cited in text.
\bibliographystyle{ieeetr} 
\bibliography{reference}

\end{document}




