% ---------------------------------------------------
% TSP 2016 Conference Template
% ---------------------------------------------------


\documentclass[conference,a4paper,twocolumn]{IEEEtran}

\IEEEoverridecommandlockouts


% *** GRAPHICS RELATED PACKAGES ***
%
\ifCLASSINFOpdf
\else
\fi


  \usepackage{tipa}
  \usepackage[pdftex]{graphicx}
	\usepackage{epsfig}
	\usepackage{epstopdf}


% *** MATH PACKAGES ***
%
\usepackage[cmex10]{amsmath}


% *** SUBFIGURE PACKAGES ***
\usepackage{subfigure}

\hyphenation{op-tical net-works semi-conduc-tor}

% *** CUSTOME (added by Alexander Senov) PACKAGES ***
\usepackage[utf8x]{inputenc} % указывает кодировку документа
\usepackage[T2A]{fontenc} % указывает внутреннюю кодировку TeX 
\usepackage[russian,english]{babel} % указывает язык документа
\usepackage{lmodern}
%\usepackage{gentium}
%\renewcommand*\rmdefault{phv}
% ---------------------------------------------------
% Begin of Document
% ---------------------------------------------------

\begin{document}

% ---------------------------------------------------
% Title
% ---------------------------------------------------

% Titles are generally capitalized except for words such as a, an, and, as,
% at, but, by, for, in, nor, of, on, or, the, to and up, which are usually
% not capitalized unless they are the first or last word of the title.
% Linebreaks \\ can be used within to get better formatting as desired.
% Do not put math or special symbols in the title.

\title{Arabic Manuscript Author Verification Using Deep Convolutional Networks}

% ---------------------------------------------------
% Authors, Affiliation, and Acknowledgment
% ---------------------------------------------------

% use a multiple column layout for up to three different
% affiliations
%\author{
%\IEEEauthorblockN{Andrei Boiarov}
%\IEEEauthorblockA{Faculty of Mathematics and Mechanics\\
%Saint-Petersburg State University\\
%Saint-Petersburg, Russia\\
%Email: andrei.boiarov@gmail.com}
%\and
%\IEEEauthorblockN{Alexander Senov}
%\IEEEauthorblockA{Faculty of Mathematics and Mechanics\\
%Saint-Petersburg State University\\
%Saint-Petersburg, Russia\\
%Email: alexander.senov@gmail.com}
%\and
%\IEEEauthorblockN{Alexander Knysh}
%\IEEEauthorblockA{Department of Near Eastern Studies\\
%University of Michigan\\
%Ann Arbor, MI 48104-1608, USA\\
%Email: alknysh@umich.edu}
%\and 
%\IEEEauthorblockN{Dmitry Shalymov}
%\IEEEauthorblockA{Faculty of Mathematics and Mechanics\\
%Saint-Petersburg State University\\
%Saint-Petersburg, Russia\\
%Email: dmitry.shalymov@gmail.com}
%\thanks{Identify applicable sponsor/s here. If no acknowledgements, delete this line.}}

% conference papers do not typically use \thanks and this command
% is locked out in conference mode. If really needed, such as for
% the acknowledgment of grants, issue a \IEEEoverridecommandlockouts
% after \documentclass

% for over three affiliations, or if they all won't fit within the width
% of the page, use this alternative format:
% 
\author{
\IEEEauthorblockN{
Andrei Boiarov\IEEEauthorrefmark{1},
Alexander Senov\IEEEauthorrefmark{2},
Alexander Knysh\IEEEauthorrefmark{3} and
Dmitry Shalymov\IEEEauthorrefmark{4}
}
\IEEEauthorblockA{
	\IEEEauthorrefmark{1}\IEEEauthorrefmark{2}\IEEEauthorrefmark{4}
	Faculty of Mathematics and Mechanics\\
	Saint Petersburg State University\\
	Saint Petersburg, Russia\\
	Email: 
		\IEEEauthorrefmark{1}a.boiarov@spbu.ru, 
		\IEEEauthorrefmark{2}alexander.senov@gmail.com, 
		\IEEEauthorrefmark{4}dmitry.shalymov@gmail.com
}
\IEEEauthorblockA{
\IEEEauthorrefmark{3}Department of Near Eastern Studies\\
University of Michigan\\
Ann Arbor, Michigan 48104-1608, USA\\
Email: alknysh@umich.edu}
\thanks{Identify applicable sponsor/s here. If no acknowledgements, delete this line.}
}

% use for special paper notices
%\IEEEspecialpapernotice{(Invited Paper)}




% make the title area
\maketitle


% ---------------------------------------------------
% Abstract
% ---------------------------------------------------

% As a general rule, do not put math, special symbols or citations
% in the abstract
\begin{abstract}
The abstract goes here. The length of the abstract should not exceed 150 words. 

Замечания
\begin{enumerate}
	\item The Method, The Data и заголовок --- "рабочие" названия
	\item Упор на authorship attribution via deep learning, без акцента на аль-Макризи
	\item По authorship attribution гугль молчит, есть схожие направления: author identification и author verification, возможно стоит отойти к ним
	\item Возможно, я неправильно форматировал список авторов, надо будет посмотреть.
	\item Предположительное распределение текста:
	\begin{itemize}
		\item 2 колонки (неполных --- на первой странице) --- Introduction
		\item 1 колонка --- The Data
		\item 2 колонки --- The Method
		\item 2 колонки --- Results and Discussion
		\item 1/2 колонки --- Conclusion
		\item 1 колонка --- Bibliography
	\end{itemize}
	Получается 8 и 1/2 - можно урезать The Method и The Data
	\item Позже надо будет удалить весь русский язык и пакет lmodern
\end{enumerate}

\end{abstract}

% no keywords

% For peer review papers, you can put extra information on the cover
% page as needed:
% \ifCLASSOPTIONpeerreview
% \begin{center} \bfseries EDICS Category: 3-BBND \end{center}
% \fi
%
% For peerreview papers, this IEEEtran command inserts a page break and
% creates the second title. It will be ignored for other modes.
\IEEEpeerreviewmaketitle

% ---------------------------------------------------
% Introduction
% ---------------------------------------------------

\section{Introduction}

\begin{enumerate}
	\item Важность задачи handwritten text visual author verification
	\item Текущее состоянее дел по handwritten text author verification
	\item Актуальность deep learning подхода к классификации изображений
	\item Акцент на том, что в authorship attribution/verification/etc deep learning не пременялся, в этом новизна
	\item Описание задачи (Problem statement): верификация авторства рукописи (изображения) посредством deep learning
	\item Описание секций
\end{enumerate}

\section{The Data}
Небольшой раздел, не более чем одной колонки.
\begin{enumerate}
	\item Почему задача верификации именно авторства аль-Макризи актуальна, цитирование Ноаха (это, в принципе, можно попросить написать Кныша)
	\item Описания данных: 
	\begin{enumerate}
		\item что за данные
		\item структура (страницы), 
		\item предобработка (обрезание и скэйлинг до 700х500), 
	\end{enumerate}
\end{enumerate}


\section{The Method}
\begin{enumerate}
	\item Описание общей структуры, желательно с блок-схемой: 
		\begin{enumerate}
			 \item Train: картинка страницы [$\to$  предобработка]$\to$ выделение патчей $\to$ тренировка сети
			 \item Application: картинка страницы [$\to$  предобработка]$\to$ выделение патчей $\to$ классификация патчей $\to$ классификация страницы
		\end{enumerate}
	Тут кроме того важно явно прописать, что мы рассматриваем authorship attribution/verification как задачу классификации
	\item Подраздел --- описание способа(способов?) выделения патчей, перевода их к размеру для сетки
	\item Подраздел --- описание сети: архитектура, обучение
	\item Подраздел --- метод определения, какому автору принадлежит документ (среднее значение вероятностей)
\end{enumerate}


\section{Results and Discussion}
\begin{enumerate}
	\item Подраздел --- результаты решения задачи классификации патчей
	\item Подраздел --- рассуждение на тему сети, визуализация скрытых слоев 
	\item Подраздел --- результаты решения задачи классификации страниц, пара картинок известных классов (о применении на Хитат в этой статье не стоит говорить, только если во 2-м разделе и заключении про "дальнейшие направлении иследований").
\end{enumerate}



% ---------------------------------------------------
% Conclusion
% ---------------------------------------------------

\section{Conclusion}
\begin{enumerate}
	\item блаблабла
 	\item сказать про компоненты связности
 	\item сказать про другие дальнейшие улучшениея --- больше данных, больше слоев
 	\item сказать про Хитат
\end{enumerate}


% ---------------------------------------------------
% References
% ---------------------------------------------------

% can use a bibliography generated by BibTeX as a .bbl file
% BibTeX documentation can be easily obtained at:
% http://www.ctan.org/tex-archive/biblio/bibtex/contrib/doc/
% The IEEEtran BibTeX style support page is at:
% http://www.michaelshell.org/tex/ieeetran/bibtex/
%\bibliographystyle{IEEEtran}
% argument is your BibTeX string definitions and bibliography database(s)
%\bibliography{IEEEabrv,../bib/paper}
%
% <OR> manually copy in the resultant .bbl file
% set second argument of \begin to the number of references
% (used to reserve space for the reference number labels box)

\begin{thebibliography}{99}

\bibitem{MBulacu} M.~Bulacu, L.~Schomaker, A.~Brink \lq\lq Text-independent writer identification and verification on offline arabic handwriting,\rq\rq~in~\emph{Proc. 9th International Conference on Document Analysis and Recognition, ICDAR}, Curitiba, 2007, pp.~769--773.

\bibitem{MBulacu} D.~Fecker, A.~Asi, W.~Pantke, V.~Märgner, J.~El-Sana, T.~Fingscheidt \lq\lq Document Writer Analysis with Rejection for Historical Arabic Manuscripts,\rq\rq~in~\emph{Proc. 14th nternational Conference on Frontiers in Handwriting Recognition, ICFHR}, Crete, 2014, pp.~743--748.

\bibitem{DL} Y.~Lecun, Y.~Bengio, G.~Hinton, \lq\lq Deep learning,\rq\rq~\emph{Nature}, no.~521, pp.~436--444, May.~2015.

\bibitem{CNN} Y.~Lecun, L.~Bottou, Y.~Bengio, P.~Haffner, \lq\lq Gradient-based learning applied to document recognition,\rq\rq~in~\emph{Proc. of the IEEE}, 1998, pp.~2278--2324.

\bibitem{Alexnet} A.~Krizhevsky,I.~Sutskever, G.~Hinton, \lq\lq ImageNet Classification with Deep Convolutional Neural Networks,\rq\rq~in~\emph{Advances in Neural Information Processing Systems}, vol.~25, 2012, pp.~1097--1105.

\bibitem{Googlenet} C.~Szegedy, W.~Liu, Y.~Jia, P.~Sermanet, S.~Reed, D.~Anguelov, D.~Erhan, V.~Vanhoucke, A.~Rabinovich, \lq\lq Going deeper with convolutions,\rq\rq~in~\emph{Proc. of the IEEE Conference on Computer Vision and Pattern Recognition}, Boston, 2015, pp.~1--9.

\bibitem{WKChen} O.~Granichin, V.~Volkovich, D.~Toledano-Kitai, \emph{Randomized Algorithms in Automatic Control and Data Mining}.	Springer-Verlag: Heidelberg, New York, Dordrecht, London, 2015, 251~p.

\bibitem{GOYoung} G.~O.~Young, \lq\lq Synthetic structure of industrial plastics (Book style with paper title and editor),\rq\rq~in~\emph{Plastics}, 2nd~ed. vol.~3, J.~Peters, Ed. New~York: McGraw-Hill, 1964, pp.~15--64.

\bibitem{WKChen} W.-K.~Chen, \emph{Linear Networks and Systems} (Book style).	Belmont, CA: Wadsworth, 1993, pp.~123--135.

\bibitem{BSmith} B.~Smith, \lq\lq An approach to graphs of linear forms (Unpublished work style),\rq\rq~unpublished.

\bibitem{EHMiller} E.~H.~Miller, \lq\lq A note on reflector arrays (Periodical style---Accepted for publication),\rq\rq~\emph{IEEE Trans. Antennas Propagat.}, to be published.

\bibitem{JWang} J.~Wang, \lq\lq Fundamentals of erbium-doped fiber amplifiers arrays (Periodical style---Submitted for publication),\rq\rq~\emph{IEEE J. Quantum Electron.}, submitted for publication.

\bibitem{JUDuncombe} J.~U.~Duncombe, \lq\lq Infrared navigation---Part I: An assessment of feasibility (Periodical style),\rq\rq~\emph{IEEE Trans. Electron Devices}, vol.~11, no.~1, pp.~34--39, Jan.~1959.

\bibitem{SPBingulac} S.~P.~Bingulac, \lq\lq On the compatibility of adaptive controllers (Published Conference Proceedings style),\rq\rq~in~\emph{Proc. 4th Annu. Allerton Conf. Circuits and Systems Theory}, New York, 1994, pp.~8--16.

\bibitem{JWilliams} J.~Williams, \lq\lq Narrow-band analyzer (Thesis or Dissertation style),\rq\rq Ph.D.~dissertation, Dept.~Elect.~Eng., Harvard~Univ., Cambridge, MA, 1993. 

\bibitem{JPWilkinson} J.~P.~Wilkinson, \lq\lq Nonlinear resonant circuit devices (Patent style),\rq\rq U.S.~Patent 3 624 12, July 16, 1990. 

\bibitem{Standard} \emph{IEEE Criteria for Class IE Electric Systems} (Standards style), IEEE Standard 308, 1969.

\bibitem{RJVidmar} R.~J.~Vidmar. (1992, August). On the use of atmospheric plasmas as electromagnetic reflectors (Online Source Style). \emph{IEEE Trans.~Plasma Sci.} [Online]. \emph{21(3)}. pp.~876-880. Available: http://www.halcyon.com/pub/journals/21ps03-vidmar

\end{thebibliography}

% ---------------------------------------------------
% End of Document
% ---------------------------------------------------

% that's all folks
\end{document}


